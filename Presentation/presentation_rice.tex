\documentclass{beamer}
% Copyright 2008  by Daina Chiba  <daina.chiba@gmail.com>
%
% This file can be redistributed and/or modified under
% the terms of the GNU Public License, version 2.

% Example presentation template file for beamerthemeRice version 0.02 (2008/11/22)

%===============================================================%
\mode<presentation> %use <handout> for handout mode
{
%\usetheme[compress, ricet, numbers]{Rice}
%\usetheme[ricet, smoothb]{Rice}
\usetheme[riceb, minimal]{Rice}
	% [ricet]		show the \Large "RICE" word mark at the top-right
	% [ricetm]		show the \large "RICE" word mark at the top-right
	% [ricets]		show the \small "RICE" word mark at the top-right
	% [riceb]		show the "RICE" word mark at the bottom-left
	% [compress]	show only the current section / subsection in the top navigation area.
	%			recommended if you have more than three subsections in at least one section
	% [minimal]	hide top navigation
	% [numbers]	show page numbers at the bottom-right
	% [noshadow]	remove shadow
	% [nologo]	remove Rice logo from the title page
	% [ricegray]	use ricegray instead of riceblue
	% [bgricegray]	use ricegray as background color
	% [bggray]	use light gray as background color
	% [smoothb]	top navigation with balls
\usefonttheme[onlymath]{serif}
	% \useoutertheme{infolines}
\setbeamercovered{transparent}
}
% In case you want to use other themes with riceblue...
%\usetheme{Frankfurt}
% AnnArbor, Antibes, Berlin, Berkeley, Bergen, Boadilla, boxes, CambridgeUS, Copenhagen
% Darmstadt, Dresden, Frankfurt
%\usecolortheme{riceowl}

\usepackage[english]{babel}
\beamerdefaultoverlayspecification{<+->}

\usepackage{mathptmx}
\usepackage{helvet}
\usepackage{courier}
% \usepackage{arev}
\usepackage[T1]{fontenc}
\usepackage{trajan}
\usepackage{animate}
\logo{\includegraphics[width=10mm,scale=0.10]{Plots/redditlogo}}
%===============================================================%
\title[Meme Propagation]{Meme Propagation Through Reddit}

\author[]{James~Chen \\ Hoik~Jang \\ Steven~Oliver \\ Adrian~Perez}

%\institute
%{
%  Department of Statistics
%}

\subject{Beamer}


\begin{document}

% Comment out the following to remove the header & footer from the title page
% \thispagestyle{empty} 

\begin{frame}
 \titlepage
\end{frame}

\begin{frame}<beamer>
  \frametitle{Outline}
  \tableofcontents
\end{frame}

\section{Background} % Introduction

\subsection{Reddit} % Reddit

\begin{frame}{Reddit: "The Front Page of the Internet"}

Reddit operates as a collection of over a million communities known as \textit{subreddits}.
These are message boards oriented around a topic of interest such as:
\begin{itemize}
\item /r/NBA
\item /r/politics
\item /r/science
\end{itemize}
Users interact within the subreddits by:
\begin{itemize}
\item posting content (images, gifs, or text discussion)
\item commenting on posts
\item voting
\end{itemize}
\end{frame}

\subsection{Memes} % Memes
\begin{frame}{Memes}

\end{frame}

\subsection{Motivation and Objectives} % Motivation
% Why is this important/interesting?
\begin{frame}{Motivation}
Reddit is one of the largest communities on the internet for new, revolutionary, and influential purposes. \\ User interaction within popular subreddits can have a great influence across the communities. Due to the richness and size of the dataset, we have proposed the following objectives:


\end{frame}

\subsection{Study Overview} % Overview
\begin{frame}{TITLE}
data size
file type
process
\end{frame}

\section{Objectives}

\subsection{TITLE}

\begin{frame}{Objectives}

\begin{block}<+->{Objective 1} 
Investigate relationships of the top 100 most active subreddits by tracking the spread of memes across those communities.
\end{block}

\begin{block}<+->{Objective 2} 
Analyze patterns of user comment activity in a multi-month span on Reddit.
\end{block}

\begin{block}<+->{Objective 3}
Determine how level of community moderator activity affects distribution of user comment scores.
\end{block}

\begin{block}<+->{Objective 4}
Analyze the fluctuations of user activity across subreddits by time of day.
\end{block}
\end{frame}

\section{Results} % Results

\subsection{Figure}

\begin{frame}{FIGURE 1 }

    \begin{figure}
      {
        \includegraphics[width=30mm,scale=0.5]{Plots/picklerick_6}
      }
    \end{figure}

\end{frame}



\section{Conclusion}

\subsection{SUBSECTION NAME}
\begin{frame}{Conclusion}
sf
\end{frame}
\end{document}
